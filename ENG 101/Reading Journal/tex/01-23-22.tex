\documentclass[journal.tex]{subfiles}

\usepackage{tikz}
\usepackage{amsmath}
\usepackage{graphicx}
\usepackage{tabularx}
\usepackage{multicol}
\usepackage{algpseudocode}
\usepackage{algorithm}

% Add vertical spacing to tables
\renewcommand{\arraystretch}{1.4}

% Begin Document
\begin{document}

\pagebreak
\section*{\today}

\subsection*{The Brief Bedford Reader: Chapter 1}

The Brief Bedford Reader, hereafter referred to as BBR, starts with the assertion that good writing is dependent on good reading.
The first chapter introduces concepts that readers use to read a work and determine several qualities of the work.
BBR then goes on to briefly utilize these concepts to introduce a student's work about a specific piece written by Nancy Mairs.
A brief outline of these concepts is as follows:

\begin{itemize}
    \item \definition{Previewing:} gather information before reading the work, such as:
    \begin{itemize}
        \item Title
        \item Author
        \item Genre
        \item Publisher
        \item Time of Publishing
    \end{itemize}

    \item \definition{Annotating:} creating notes about the work
    \item \definition{Summarizing:} allows the reader to ensure a grasp of the work
    
    \item \definition{Thinking Critically}
    \begin{itemize}
        \item Analyze
        \item Infer
        \item Synthesize
        \item Evaluate
    \end{itemize}
    
    \item \definition{Analysis of Written Works}
    \begin{itemize}
        \item Meaning: Thesis, Purpose
        \item Writing Strategy: Audience, Method, Evidence, Structure
        \item Language: Tone, Word Choice, Imagery
    \end{itemize}
\end{itemize}

BBR then extends the definition of a Text to include other mediums, such as the example photo.
The same concepts outlined above should be applied to all works. 

Applying these concepts to the first chapter results in an interesting thought: my natural instinct was to read the work of Mairs that is presented under the light of the above concepts.
However, that is contradictory to the intent of the first chapter.
The presented work is simply a demonstration, and the authors apply the concepts and present their own results.


\subsection*{\textit{Camping Out} by Ernest Hemingway}

Hemingway is known to be one of the great writers of the 20th century, and the preamble to this article points out that while this work was written before his novels it is a clear demonstration of his style.
The work's primary intent is instructional advice for being able to spend a week in the bush, for the person who has not used it as a vacation prior.
The work imparts a lot of knowledge, written in a way that implies that the writer has experienced this firsthand.
The writing is a product of its time as well, not only through references to different products, but through little sayings and societal ideologies like the quip of an average man being at least as capable as his wife.
It's written in a light style, as if spoken informally between an experienced woodsman and a rookie. 



\end{document}