\documentclass[12pt]{article}

\usepackage{tikz}
\usepackage{amsmath}
\usepackage{graphicx}
\usepackage{tabularx}
\usepackage{multicol}
\usepackage{algpseudocode}
\usepackage{algorithm}

% Geometry 
\usepackage{geometry}
\geometry{letterpaper, left=15mm, top=20mm, right=15mm, bottom=20mm}

% Fancy Header
\usepackage{fancyhdr}
\renewcommand{\footrulewidth}{0.4pt}
\pagestyle{fancy}
\fancyhf{}
\chead{CSC 420 - Artificial Intelligence}
\lfoot{CALU Spring 2022}
\rfoot{RDK}

% Add vertical spacing to tables
\renewcommand{\arraystretch}{1.4}

% Macros
\newcommand{\definition}[1]{\underline{\textbf{#1}}}

\newenvironment{rcases}
  {\left.\begin{aligned}}
  {\end{aligned}\right\rbrace}

% Begin Document
\begin{document}

\section*{Notes, Chapter 2}

\subsection*{Agents and Environments}

\begin{itemize}
    \item An agent is anything that can be viewed as perceiving its environment through sensors and acting upon that environment through actuators
    \item Agents include humans, robots, vehicles, etc
    \item The agent function maps from percept histories to actions $ f: P^* \to A$
    \item The agent program runs on the physical architecture to produce $f$ \\ $agent = architecture + program$
\end{itemize}

The goal of AI then is to link the percepts of the environment to actions that it can take.


\subsection*{Rationality}

\begin{itemize}
    \item A \definition{rational agent} does the right thing, but what does it mean to do the right thing?
    \item A \definition{performance measure} to evaluate the behavior of the agent in an environment
    \begin{itemize}
        \item One point per square cleaned up in time T?
        \item One point per clear square per time step, minus one per move?
    \end{itemize}
    \item A rational agent chooses whichever action maximizes the expected value of the performane measure given the percept sequence to date.
    \item What is rational at any given time depends on four things:
    \begin{enumerate}
        \item The performance measure that defines the criterion of success
        \item The agent's prior knowledge of the environment
        \item The actions that the agent can perform
        \item The agent's percept sequence to date
    \end{enumerate}

\end{itemize}

\subsubsection*{Definition of a Rational Agent}

For each possible percept sequence, a rational agent should select an action that is expected to maximize its performance
measure given the evidence provided by the percept sequence, and whatever built-in knowledge the agent has.

\subsection*{The Nature of Environments / Task Environments (PEAS)}

To design a rational agent, we must specify the task environment:
\begin{itemize}
    \item Performance measure
    \item Environment
    \item Actuators
    \item Sensors
\end{itemize}

\subsubsection*{PEAS - Example - Automated Taxi}

\begin{itemize}
    \item Performance Measure: Profit, Safety, Destination (minimal path), Comfort
    \item Environment: US streets/freeways, traffic, weather
    \item Actuators: Steering, Accelerator, Brake, Horn, Speakers/Display, etc
    \item Sensors: Cameras, Accelerometers, Engine Sensors, GPS, etc
\end{itemize}


\subsection*{Properties of Task Environments}

\begin{itemize}
    \item Fully Observable vs Partially Observable
    
    \item Single-agent vs Multi-agent
    \begin{itemize}
        \item Competitive vs Cooperative environment
    \end{itemize}
    
    \item Deterministic vs Nondeterministic
    
    \begin{itemize}
        \item In deterministic environments, the next state of the environment is completely determined by the current state and the action executed by the agent
    \end{itemize}
    
    \item Episodic vs Sequential
    \begin{itemize}
        \item In an episodic environment, the agen'ts experience is divided into atomic episodes. 
        In each episode, the agent receives a percept and then performs a single action. 
        The next episode does not depend on the action taken in the previous ones.
        
        \item In a sequential environment, the current decisions could affect the future decisions.
    \end{itemize}

    \item Static vs Dynamic: a dynamic environment can be changed for the agent

    \item Discrete vs Continuous: Able to process a snapshot vs ongoing inputs

\end{itemize}

\begin{tabular}{l | l l}
    Task Environment & Self-Driving Taxi & Crossword \\ \hline
    Observable & Partially & Fully \\
    Agents & Multi & Single \\
    Deterministic & No & Yes \\
    Episodic/Sequential & Sequential & Sequential \\ 
    Static/Dynamic & Dynamic & Static \\
    Continuous/Discrete & Continuous & Discrete 
\end{tabular}

\subsection{Agent Types}

Any type of agent can have a learning model added to it.

\subsubsection{Simple Reflex Agents}

This agent type selects actions based on the currect percept, ignoring the rest of the percept history.

\subsubsection{Model-based reflex agents}

This agent maintains some sort of internal state that depends on the percept history.
This is to handle partial observability to let the agent keep track of the part of the world it cannot see now.

\subsubsection{Goal-based Agents}

This type keeps track of the world state as well as a set of goals it is trying to achieve and chooses an action that will (eventually) leadd to the achievements of its goals.

\subsubsection{Utility-based Agents}

This type can be used wwhen there are conflicting goals. It uses a model of the world, along with a utility function that measures its performance among the states of the world.
Then it chooses the action that maximizes the expected utility.
Goal based agents and utility-based agents are more flexible than simple reflex and model-based reflex agents.
Why?



\end{document}