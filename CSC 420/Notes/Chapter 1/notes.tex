\documentclass[12pt]{article}

\usepackage{tikz}
\usepackage{amsmath}
\usepackage{graphicx}
\usepackage{tabularx}
\usepackage{multicol}
\usepackage{algpseudocode}
\usepackage{algorithm}

% Geometry 
\usepackage{geometry}
\geometry{letterpaper, left=15mm, top=20mm, right=15mm, bottom=20mm}

% Fancy Header
\usepackage{fancyhdr}
\renewcommand{\footrulewidth}{0.4pt}
\pagestyle{fancy}
\fancyhf{}
\chead{CSC 420 - Artificial Intelligence}
\lfoot{CALU Spring 2022}
\rfoot{RDK}

% Add vertical spacing to tables
\renewcommand{\arraystretch}{1.4}

% Macros
\newcommand{\definition}[1]{\underline{\textbf{#1}}}

\newenvironment{rcases}
  {\left.\begin{aligned}}
  {\end{aligned}\right\rbrace}

% Begin Document
\begin{document}

\section*{Notes, Chapter 1}

\subsection*{Definition of AI}

\begin{itemize}
    \item ``The art of creating machines that perform functions that require intelligence when performed by people.'' - Kurzweil, 1990

    \item ``The branch of computer science that is concerned with the automation of intelligent behavior.'' - Luger and Stublefield
\end{itemize}


\subsection*{Different Types of AI}

\subsubsection*{Focuses}
\begin{itemize}
    \item Thought processes and reasoning
    \item Intelligent Behavior - intelligent acts, such as a robot performing a task correctly
\end{itemize}

\subsubsection*{Two Dimensions of AI}
\begin{itemize}
    \item \definition{Human vs Rational:} Does an AI emulate human patterns and irrationality?
    \item \definition{Thought vs Behavior:} Does it think or act?
\end{itemize}

\subsubsection*{Types of AI}
\begin{itemize}
    \item Systems that think like Humans
    \item Systems that think rationally
    \item Systems that act like Humans
    \item Systems that act rationally
\end{itemize}



\subsection*{Acting Humanly: The Turing Test}

A computer passes the test of a human tester, after posing some written questions, 
if the human cannot tell whether the written responses come from a person or from a computer.

A machine to model a human needs several capabilities:
\begin{itemize}
    \item \definition{Natural Language Processing:} to communicate successfully in a human language
    \item \definition{Knowledge Representation:} to store what it knows or hears
    \item \definition{Automated Reasoning:} to answer questions and draw a new conclusion
    \item \definition{Machine Learning:} to adapt to new situations and to detect patterns
\end{itemize}

A \definition{Total Turing Test} requires interactions with the real world, necessitating additional requirements:
\begin{itemize}
    \item Computer vision and speech recognition to perceive the world
    \item Robotics to manipulate objects and move about
\end{itemize}


\subsection*{Thinking Humanly: Cognitive Modelling}

Cognitive Sciences is an interdisciplinary field consisting of fields like Artificial Intelligence, Psychology, Linguistics,
Philosophy, and Anthropology that tries to form theories of human behavior and reasoning.

\subsection*{Thikning Rationally: Laws of Thought}

Reasoning using a mathematical model:
\begin{itemize}
    \item \definition{Logic:} encodes knowledge in formal logical statements and use mathematical deduction to perform reasoning
    \item \definition{Probability:} the theory of Probability allows reasoning with uncertain information
\end{itemize}

\subsection*{Acting Rationally: Rational Agents}

\begin{itemize}
    \item An agent is an entity that perceives its environment and is able to execute actions to change it
    \item Agents have inherent goals that they want to achieve
    \item A rational agent acts in a way to maximize the achievement of its goals
    \item True maximixation of goals requires techniques from multiple science and unlimited computational abilities
    \item Limited rationality involves maximizing goals within the computational and other resources available
\end{itemize}


\end{document}