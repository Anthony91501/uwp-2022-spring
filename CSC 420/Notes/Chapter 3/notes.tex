\documentclass[12pt]{article}

\title{Chapter 3}
\author{}
\date{}

\usepackage{tikz}
\usepackage{amsmath}
\usepackage{graphicx}
\usepackage{tabularx}
\usepackage{multicol}
\usepackage{algpseudocode}
\usepackage{algorithm}

% Geometry 
\usepackage{geometry}
\geometry{letterpaper, left=15mm, top=20mm, right=15mm, bottom=20mm}

% Fancy Header
\usepackage{fancyhdr}
\renewcommand{\footrulewidth}{0.4pt}
\pagestyle{fancy}
\fancyhf{}
\chead{CSC 420 - Artificial Intelligence}
\lfoot{CALU Spring 2022}
\rfoot{RDK}

% Add vertical spacing to tables
\renewcommand{\arraystretch}{1.4}

% Macros
\newcommand{\definition}[1]{\underline{\textbf{#1}}}

\newenvironment{rcases}
  {\left.\begin{aligned}}
  {\end{aligned}\right\rbrace}

% Begin Document
\begin{document}

\maketitle

\section{Problem Solving Process}

\begin{enumerate}
    \item Goal forumlation
    \item Problem formulation
    \item Search
    \item Execution
\end{enumerate}

\section{Search Problems and Solutions}

A search problem can be defined formally as follows:
\begin{itemize}
    \item A set of \textbf{all states}, called the \textbf{state space}
    \item \textbf{initial state}
    \item One or more \textbf{goal states}
    \item The \textbf{actions}
    \begin{itemize}
        \item Given a state $s$, a function $ACTIONS(s)$ returns a finite set of actions that can be executed in $s$
    \end{itemize}
    \item A \textbf{transition model}, describes what each action does
    \begin{itemize}
        \item $RESULT(s,a)$ returns the state that results from doing action $a$ in state $s$
    \end{itemize}
    \item An \textbf{action cost function}, denoted by $ACTION-COST(s, a, s')$. It gives the numeric cost of applying action $a$ in state $s$ to reach state $s'$.
\end{itemize}

A search algorithm can be conducted to find:
\begin{itemize}
    \item A \textbf{solution} is a path of actions sequence from the initial state to the goal state
    \item An \textbf{optimal solution} is the lowest path cost among all students
\end{itemize}

\section{Formulating Problems}

\begin{itemize}
    \item When formulating a problem, we are creating a \textbf{model}
    \item A \textbf{model} is an abstract mathematical description and not a real thing
    \item \textbf{Abstraction} is the process of removing details from a presentation
    \item One issue is finding a suitable \textbf{Level of Abstraction}
\end{itemize}

\subsection{Example Problems}

There are two types of problems:
\begin{enumerate}
    \item \textbf{Standardized problems} are intended to illustrate or exercise different problem-solving methods and used as a benchmark
    \item \textbf{Real-world problems}
\end{enumerate}

\subsubsection{Real-World Problems}

\definition{Routing-finding problems}: a set of locations that must be visited to reach a single destination

Examples:
\begin{itemize}
    \item Driving directions
    \item Routing video streams over the Internet
    \item Military operations planning
    \item Airline traveling-planning systems
\end{itemize}

Airline Traveling-Planning website example:
\begin{itemize}
    \item States: Each state includes a location (eg, an airport) and the current time
    \item Initial State: the user's home airport
    \item Actions: take any flight from the current location after the current time, in any seat class, leaving enough time for within airport transfer if needed
    \item Transition Model: the state resulting from taking a flight
    \item Goal State: A destination city (goal can be more complex, such as only a nonstop flight)
    \item Action Cost: A combination of cost, waiting time, flight time, seat quality, time of day, etc
\end{itemize}

\definition{Touring problems}: a set of locations that must be visited, rather than a single destination

Examples:
\begin{itemize}
    \item Routing of a school bus
    \item VLSI layout: cell layout and channel Routing
    \item Robot Navigation
    \item Automatic assembly sequencing
\end{itemize}



\end{document}
