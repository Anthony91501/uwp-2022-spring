\documentclass[12pt]{article}

\usepackage{tikz}
\usepackage{amsmath}
\usepackage{graphicx}
\usepackage{tabularx}
\usepackage{multicol}
\usepackage{algpseudocode}
\usepackage{algorithm}
\usepackage{setspace}

% Geometry 
\usepackage{geometry}
\geometry{letterpaper, left=15mm, top=20mm, right=15mm, bottom=20mm}

% Fancy Header
\usepackage{fancyhdr}
\renewcommand{\footrulewidth}{0.4pt}
\pagestyle{fancy}
\fancyhf{}
\chead{MAT 381 - Calculus 3}
\lfoot{CALU Spring 2021}
\rfoot{RDK}

% Add vertical spacing to tables
\renewcommand{\arraystretch}{1.4}

\onehalfspacing

% Macros
\newcommand{\definition}[1]{\underline{\textbf{#1}}}

\newenvironment{rcases}
  {\left.\begin{aligned}}
  {\end{aligned}\right\rbrace}

% Begin Document
\begin{document}

\section*{8.9 Improper Intgerals}

There are typically two types of improper integrals:

\begin{enumerate}

    \item The interval of integration is infinite \\
    $ \int_{1}^{\infty} \frac{1}{x^2} \,dx $

    \item The integrand is unbounded on the interval of integration \\
    $ \int_{-1}^{1} \frac{1}{x} \,dx $

\end{enumerate}

How do you find the integral $\int_{1}^{\infty} \frac{1}{x^2} \,dx$?

Consider the integral $\int_{1}^{b} \frac{1}{x^2} \,dx$, where $b > 1$ is an arbitrary real number.

Then we can write $\lim_{b\to\infty} \int_{1}^{b} \frac{1}{x^2} \,dx $

Changing the infinity to a variable, and then taking the limit of the integral to infinity allows for a possible solution.

\subsubsection*{Examples}

\begin{itemize}

    \item $ \int_{0}^{\infty} \frac{sin(x)}{x} \,dx = \lim_{b\to\infty} \int_{0}^{b} \frac{sin(x)}{x} \,dx $

    \item $ \int_{-\infty}^{1} \frac{e^x}{x^2} \,dx = \lim{a\to\infty} \int_{a}^{1} \frac{e^x}{x^2} \,dx $

\end{itemize}


\subsubsection*{Example}

Find, if possible: $ \int_{1}^{\infty} \frac{1}{x^2} \,dx $

\begin{enumerate}

    \item Rewrite as a limit \\ $\lim_{b\to\infty} \int_{1}^{b} \frac{1}{x^2} \,dx$

    \item Solve the integral first: \\
    \begin{equation*}
        \int_{1}^{b} \frac{1}{x^2} \,dx \\ 
        = -\frac{1}{x} \bigg\rvert_{1}^{b} \\
        = -\frac{1}{b} + 1 
    \end{equation*}

    \item Then use that to find the limit: \\
    \begin{equation*}
        \lim_{b\to\infty} -\frac{1}{b} + 1 = -\frac{1}{\infty} + 1 = 0 + 1 = 1
    \end{equation*}

    \item Thus $ \int_{1}^{\infty} \frac{1}{x^2} \,dx = 1 $

\end{enumerate}

\begin{itemize}
    \item An interval is said to \definition{converge} when it goes to a finite value, and \definition{diverge} when it evaluates to an infinite value.
\end{itemize}


\subsection*{Examples, Infinite Integrals}

Evaluate the following integrals, or state if they if diverge.

\begin{enumerate}

    \item \begin{flalign*}
        &\int_{2}^{\infty} \frac{1}{x} \,dx &&\\
        &= \lim_{b\to\infty} \int_{2}^{b} \frac{1}{x} \,dx &&\\
        &= \lim_{b\to\infty}( \ln(x) \bigg\rvert_{2}^{b} ) &&\\
        &= \lim_{b\to\infty}( \ln(b) - \ln(2) ) &&\\
        &= \ln(\infty) - \ln(2) = \infty &&
    \end{flalign*}
    Therefore, the integral diverges.

    \item \begin{flalign*}
        &\int_{-\infty}^{0} e^{2x} \,dx &&\\
        &= \lim_{a\to-\infty} \int_{-a}{0} e^{2x} \,dx &&\\
        &= \lim_{a\to-\infty} (\frac{1}{2} e^{2x} \bigg\rvert_{a}^{0}) &&\\
        &= \lim_{a\to-\infty} ( \frac{1}{2} - \frac{1}{2e^{2a}} ) &&\\
        &= \frac{1}{2} - \frac{1}{2e^{2\infty}} &&\\
        &= \frac{1}{2} &&
    \end{flalign*}
    Therefore, the integral converges.

    \item \begin{flalign*}
        & \int_{-\infty}^{\infty} e^{4x} \,dx &&\\
        & = \int_{-\infty}^{0} e^{4x} \,dx + \int_{0}^{\infty} e^{4x} \,dx &&\\
        & \int_{-\infty}^{0} e^{4x} \,dx = \lim_{a\to\-\infty} ( \frac{1}{4} e^{4x} \bigg\rvert_{a}^{0}) &&\\
        & = \lim_{a\to-\infty} (\frac{1}{4} - \frac{1}{4}e^{4a}) = \frac{1}{4} &&\\
        & \int_{0}^{\infty} e^{4x} \,dx = \lim_{b\to\infty} ( \frac{1}{4} e^{4x} \bigg\rvert_{0}^{b}) &&\\
        & = \lim_{b\to\infty} (\frac{1}{4}e^{4b} - \frac{1}{4}) = \infty &&\\
        & = \int_{-\infty}^{0} e^{4x} \,dx + \int_{0}^{\infty} e^{4x} \,dx = \frac{1}{4} + \infty = \infty&&\\
    \end{flalign*}
    Therefore, the integral diverges.

\end{enumerate}




\end{document}